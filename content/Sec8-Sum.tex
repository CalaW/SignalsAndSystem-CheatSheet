\section{总结}

这一学期的课程让我收获良多。但由于参与了2022北京冬奥会及冬残奥会,我在4月6日才解除隔离返回学校。赛事服务期间的课程由于时间可能会产生冲突,我都是利用学堂在线的MOOC课程进行学习的。在学习过程中,我感觉MOOC的内容编排上提前介绍的内容太多,不是很“循序渐进”。比如第一章的系统分类,一下子摆出了所有的分类类型,一时间信息量有些过于庞大。我认为郑教授的书的处理方法会比较适合自学,第一章先介绍一小部分系统分类,剩下的留到第二章线性时不变系统的时候统一介绍。或许MOOC因为回放方便的特性,会被认为比较适合在短时间内讲述结构性强,知识密度高的内容,但我认为MOOC另一个重要的方面是方便同学们进行自学,也有必要考虑知识点引入的循序渐进。以上是我关于MOOC课程的一些小想法。

另外由于4月6日前未能参与课堂教学环节(有学校的假条),还得麻烦卓大大留意一下我的课堂参与分数。我也会在考试后单独联系您,看看怎么处理这一部分分数~